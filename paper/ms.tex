\documentclass[useAMS, usenatbib, preprint, 12pt]{aastex}
\usepackage{cite, natbib}
\usepackage{float}
\usepackage{epsfig}
\usepackage{cases}
\usepackage[section]{placeins}
\usepackage{graphicx, subfigure}
\usepackage{color}
\usepackage{bm}

\newcommand{\columbia}{1}
\newcommand{\simons}{2}
\newcommand{\nyu}{3}
\newcommand{\cca}{4}
\newcommand{\mpia}{5}
\newcommand{\cds}{6}
\newcommand{\naigrain}{333}
\newcommand{\nmcquillan}{100}
\newcommand{\kepexample}{1430163}
\newcommand{\kepexampleperiod}{4}
\newcommand{\aigrainexampleperiod}{20.8}
\newcommand{\Kepler}{{\it Kepler}}
\newcommand{\kepler}{\Kepler}
\newcommand{\corot}{{\it CoRoT}}
\newcommand{\Ktwo}{{\it K2}}
\newcommand{\ktwo}{\Ktwo}
\newcommand{\TESS}{{\it TESS}}
\newcommand{\LSST}{{\it LSST}}
\newcommand{\Wfirst}{{\it Wfirst}}
\newcommand{\SDSS}{{\it SDSS}}
\newcommand{\PLATO}{{\it PLATO}}
\newcommand{\Gaia}{{\it Gaia}}
\newcommand{\gaia}{{\it Gaia}}
\newcommand{\Teff}{$T_{\mathrm{eff}}$}
\newcommand{\teff}{$T_{\mathrm{eff}}$}
\newcommand{\FeH}{[Fe/H]}
\newcommand{\feh}{[Fe/H]}
\newcommand{\ie}{{\it i.e.}}
\newcommand{\eg}{{\it e.g.}}
\newcommand{\logg}{log \emph{g}}
\newcommand{\dnu}{$\Delta \nu$}
\newcommand{\numax}{$\nu_{\mathrm{max}}$}
\newcommand{\acfRMS}{9.27}
\newcommand{\pgramRMS}{1.25}
\newcommand{\mcmcRMS}{0.46}
\newcommand{\racomment}[1]{{\color{red}#1}}

\begin{document}

\title{How useful is gyrochronology?}

\author{Ruth Angus\altaffilmark{\columbia, }\altaffilmark{\simons}}

\altaffiltext{\columbia}{Department of Astronomy, Columbia
University, NY, NY}
\altaffiltext{\simons}{Simons Fellow, RuthAngus@gmail.com}


\begin{abstract}
\end{abstract}

\include{intro}

\section{What makes gyrochronology difficult?}
Binaries, cosmic variance, complicated model.

\section{Competing models}
Van Saders, Barnes 2007, 2010, Matt 2012, etc.

\section{Precision and Accuracy of gyrochronology}
Cluster and binary test.

\section{Future potential}
Missions.

\section{Conclusion}

% acknowledgements
This research was funded by the Simons Foundation.
Some of the data presented in this paper were obtained from the Mikulski
Archive for Space Telescopes (MAST).
STScI is operated by the Association of Universities for Research in
Astronomy, Inc., under NASA contract NAS5-26555.
Support for MAST for non-HST data is provided by the NASA Office of Space
Science via grant NNX09AF08G and by other grants and contracts.
This paper includes data collected by the Kepler mission. Funding for the
Kepler mission is provided by the NASA Science Mission directorate.

\bibliographystyle{plainnat}
\bibliography{gyroscope}
\end{document}
